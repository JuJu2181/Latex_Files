%comments in this code are either test codes or guidance for future use
%Coded By : Anish Shilpakar (CRN:KCE075BCT008)
\documentclass[12 pt,a4paper]{report}
\usepackage{amsmath,amssymb,esint}

\begin{document}
\chapter{Typesetting Mathematics in \LaTeX}
\begin{equation}
ax+by+c=1
\end{equation}\\
\begin{equation}
X+y
\end{equation}
\section{Simple Equations}
$$Y_{D}=Z_{C}$$
$$x+y=1$$
$$ax+by+c=z+1$$
\pagebreak
\begin{eqnarray*}
\\\\
Y_{D}&=&Z_{C}\\
x+y&=&1\\
ax+by+c&=&z+1\\
x^{10^{15}}&=&y^{12}
\theta\\
\beta\\
\Omega\\
\rho\\
\alpha\\
\delta\\
\Delta\\
\chi\\
\phi\\
\Pi\\
\pi\\
\gamma\\
\Gamma\\
\lambda\\
l*b\\
a.b\\
Ohm'slaw:\\
I=\frac{V}{R}
\end{eqnarray*}\\
Ohm's law:\\
$$I=\frac{V}{R}$$
\section{Square Root, Cube Root, fourth root and nth Root}
$$\sqrt{[n]x}$$
$$\sqrt[3]{8}$$
$$\frac{1}{2}$$
$$\sqrt{2},\sqrt[3]{81},\sqrt[4]{16},\sqrt[n]{x}$$
Evaluate:
$$\sqrt{2+\sqrt{2+\sqrt{2+}...}}$$
\section{Simple Equations}
$$ax+by+cz=0$$
$$\frac{a}{b}=\frac{c}{d}$$
$$\frac{a^{x}}{y}=b^{y}.c^{x}$$
\section{Diode Equations:}
\begin{eqnarray}
i_D=I_S(e^{\frac{v_D}{\eta.V_T}}-1)\\
V_{D2}-V_{D1}=2.303\eta V_{T}ln(\frac{I_{D2}}{I_{D1}})\\
i_D=I_S\left(^{\frac{V_D}{\eta.V_T}}-1\right)\\
V_{D_{2}}-V_{D_{1}}=2.303\eta V_{T}\ln(\frac{I_{D2}}{I_{D1}})\\
\end{eqnarray}
\chapter{Calculas}
\section{Differential Equations:}
\begin{eqnarray}
\frac{dy}{dx}+1=0&\\
\frac{d^{2}y}{dx^{2}}+\frac{dy}{dx}+6=0&\\
\frac{d^{2}y}{dx^{2}}+5\frac{dy}{dx}+6=1&\\
\frac{d^{2}y}{dx^{2}}+5\frac{dy}{dx}+6=2e^{ax}\sin{bx}&\\
\frac{d^{3}y}{dx^{3}}+8\frac{d^{2}y}{dx^{2}}+5\frac{dy}{dx}+6=2e^{ax}\cos{bx}&\\
\frac{d^{2}y}{dx^{2}}=-\frac{y^{3}}{x^{3}}\\
\sin^{2}x+\cos^{2}x=1
\end{eqnarray}
\pagebreak
\section{Integration:}
\subsection{Line Integrals:}
\begin{eqnarray}
\int\\
\int f(x)dx\\
\int_{a}^{b} f(x)dx\\
\int_{a}^{b} f(x)dx\\
\int_{x_{1}}^{x_{2}} f(x)dx 
\end{eqnarray}
\subsection{Surface Integrals: }
\begin{eqnarray}
\iint_{S} f(x,y)dx\,dy\\
\oiint_{S} f(x,y)dx\,dy\\
\int_{x_{1}}^{x_{2}}\int_{y_{1}}^{y_{2}}f(x,y)dx\,dy
\end{eqnarray}
\subsection{Volume Integral: }
\begin{equation}
\iiint_{V}f(x,y,z)dx\,dy\,dz
\end{equation}
\begin{equation}
\int_{x_{1}}^{x_{2}}\int_{y_{1}}^{y_{2}}\int_{z_{1}}^{z_{2}} f(x,y,z)dx\,dy\,dz
\end{equation}
\section{Fourier Integral:}
\begin{equation}
f(t)=\frac{1}{2\pi}\int_{-\infty}^{+\infty}F(S)e^{jwt}dt
\end{equation}
\section{Fourier Transform:}
\begin{equation}
F(S)=\int_{-\infty}^{+\infty}f(t)e^{-jwt}dt
\end{equation}
\section{Inverse Laplace Transform:}
\begin{equation}
f(t)=\frac{1}{2\pi}\int_{-\infty}^{+\infty}F(S)e^{st}dt
\end{equation}
\section{Laplace Transform:}
\begin{equation}
F(S)=\int_{-\infty}^{+\infty}f(t)e^{-st}dt
\end{equation}
\section{Maxwell's Equations:}
\subsection{Differential form of Maxwell's Equation:}
Differential form of Maxwell's equations are categorized as follows:\\
Gauss law:
\begin{eqnarray*}
\overset{\rightarrow}{\nabla}.\overset{\rightarrow}{E}=\frac{\rho}{\epsilon_{0}}\\
\overset{\rightarrow}{\nabla}.\overset{\rightarrow}{E}=\frac{\rho}{\epsilon_{0}}\\
\overset{\rightarrow}{\nabla}.\overset{\rightarrow}{E}=\frac{\rho}{\epsilon_{0}}\\
%\overset{\rightarrow}{a}.\overset{\rightarrow}{b}=0\\
%\overset{\rightarrow}{a}\times\overset{\rightarrow}{b}=0
\end{eqnarray*}
$$\overset{\rightarrow}{a}.\overset{\rightarrow}{b}=0$$
$$\overset{\rightarrow}{a}\times\overset{\rightarrow}{b}=0$$
$$\overline{a}$$
\begin{center}
\underline{Electrical Engineering}
\end{center}
%\,\,\,\,\,\,\,\,Gauss law for magnetism:

Gauss law for magnetism:
$$\overset{\rightarrow}{\nabla}.\overset{\rightarrow}{B}=0$$
Faraday's law of induction:
$$\overset{\rightarrow}{\nabla}\times\overset{\rightarrow}{E}=-\frac{B}{\delta t}$$
Ampere's Circuital Law:
$$\overset{\rightarrow}{\nabla}\times\overset{\rightarrow}{E}=-\mu_{o}(J+\epsilon_{o}\frac{dE}{dt})$$
\subsection{Integral form of Maxwell's Equation}
Integral Form of Maxwell's equation are categorized as follows:\\
Gauss Law:\\
$$\int\overset{\rightarrow}{E}.\overset{\rightarrow}{dl}=0$$ 
Gauss Law for Magnetism:\\
$$\int_{S}\overset{\rightarrow}{B}.\overset{\rightarrow}{ds}=0$$
\section{Typesetting Matrices and Determinants:}
$$
A=
\left(
\begin{array}{ccc}
a & b & c\\
d & e & f\\
g & h & i
\end{array}
\right)
$$\\
$$
B=
\left[
\begin{array}{ccc}
1 & 2 & 3\\
4 & 5 & 6\\
7 & 8 & 9
\end{array}
\right]
$$\\
$$
C=
\left\lbrace
\begin{array}{ccc}
\alpha & \beta & \gamma\\
\varsigma & \epsilon & \kappa\\
\Delta & \lambda & \hbar
\end{array}
\right\rbrace
$$
$$
A=              
\left[
\begin{array}{ccc}
a_{11} & a_{12} & a_{13}\\
a_{21} & a_{22} & a_{23}\\
a_{31} & a_{32} & a{33}
\end{array}
\right]
=
B=
\left[
\begin{array}{ccc}
x_{11} & x_{12} & x_{13}\\
x_{21} & x_{22} & x_{23}\\
x_{31} & x_{32} & x{33}
\end{array}
\right]
$$

%\begin{flushright}
\begin{eqnarray}
\,\,\,\,\,\,\,\,\,\,x+y-z=5\\
x-y+z=5\\
x-z=0
\end{eqnarray}
%\end{flushright}

%\begin{align}
%x+y-z=5\\
%x-y+z=5\\
%x-z=0
%\end{align}

\begin{eqnarray*}
\,\,\left|
\begin{array}{cc}
a & b\\
c & d\\
\end{array}
\right|=\,ad-bc
\end{eqnarray*}

\begin{eqnarray}   %Found error here
a_{1}x+b_{1}y+c_{1}z=d_{1}\\
a_{2}x+b_{2}y+c_{2}z=d_{2}\\
a_{3}x+b_{3}y+c_{3}z=d_{3}
\end{eqnarray}
\section{Equations involving limits:}
$$\underset{\Delta x\rightarrow 0}{\lim}\,\frac{f(x+\delta x)-f(x)}{\Delta x}$$
$$\underset{x\rightarrow 0}{\lim}\,\frac{\sin x}{x}=1$$
\section{Typesetting Boolean Expressions}
$$\underline{x+y}=\overline{x}.\overline{y}$$
$$\overline{A}BC+A\overline{BC}+\overline{ABC}$$
\section{Typesetting equations involving summa-tion}
$$\sum_{k=0}^{\infty}\frac{(-1)^{k}}{k+1}=\int_{0}^{1}\frac{dx}{1+x}$$
\end{document}

