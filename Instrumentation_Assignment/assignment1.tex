\documentclass[12pt]{article}
\usepackage[a4paper,top=1in,bottom=1in,left=1.5in,right=1.5in]{geometry}
\begin{document}
\begin{center}
\begin{Large}
\textbf{Assignment 1}\\\
\end{Large}
\begin{large}
\textbf{Submitted By }\\
\end{large}
Anish Shilpakar\\
CRN : KCE075BCT008\\
\end{center}

\section{Direct And Indirect Measurement}
Measurement basically refers to the comparison of an unknown quantity with a predefined standard quantity. Measurement plays a significant role in instrumentation system and is used for data management and manipulation in various fields of science and technology.
Based on the ways of performing dimensional measurement, it can be classified into two types i-e 
\begin{enumerate}
\item Direct Measurement 
\item Indirect Measurement
\end{enumerate} 
\subsection{Direct Measurement}
If the unknown quantity is compared to a primary or secondary standard, then the measurement is referred as direct measurement.  Here we compare the unknown quantity directly to the standard scale and scale is expressed as a numerical number and a unit. The direct method of measurement is also known as absolute measurement and it can be performed over a wide range specified by the scale of measuring instrument, but there is also possibility for the measurement to be wrong due to erroneous readings of the scale. Here the errors does not occur because of the error in standards, but because of the human limitations in taking the readings. So the results produced by direct comparison method of measurement are not always accurate. 

For example, the measuring instruments such as Vernier callipers, micrometers, measuring tapes and other coordinate measuring instruments are used for direct measurement e.t.c. 

\subsection{Indirect Measurement}
In the indirect method of measurement the unknown quantity is converted into some other measurable quantity and then that measurable quantity is measured. The indirect measurement is used for measuring those quantities which cannot be measured directly using some instruments. It is also known as comparative measurement as the comparison is performed using object with standard dimensions.The more predetermined that the shape and dimensions of a reference device are, the easier the measurement becomes. 

For example, As strain in the bar cannot be measured using applied force directly we measure it in terms of  the electrical resistance of the bar., gauge blocks, ring gauges e.t.c.

\pagebreak
\section{Analog And Digital Signals}
Analog and Digital signals are the forms of electric signals that are used to transmit information. 

\subsection{Analog Signal}
An analog signal is a continuous signal or wave that changes over time usually represented by a sine wave and described by the amplitude, frequency and phase. There may be a range for the values that the signal can take, but within this range the signal could represent any value. Unlike the digital signal it has infinite resolution. It records the analog waveforms in the same form. During data transmission the analog signals are likely to be deteriorated by noise and read/write cycle and are also more prone to distortion. The analog signal processing can be done in real time and it consumes less bandwidth and is also cheaper and portable. 

For example, human voice in air is an analog signal \\

The instruments that uses the analog signals to display the quantity under measurement are known as analog instruments. These devices are less flexible and draw large amount of power. These devices usually have a scale which is cramped at the lower end and may give considerable observational errors.

For example, thermometer

\subsection{Digital Signal}
Digital signals are discrete time signals generated by digital modulation which carries information in binary form. These signals are generally represented by square waves and are described by bit rate and bit intervals. These signals have finite resolution which depends on the number of bits used to convey data. Digital signals samples analog waveforms into a limited set of numbers and then records them. These signals can be noise immune without deterioration during transmission and read/write cycle and is also less prone to distortion. There is no guarantee that the digital signal processing can be done in real time and consumes more bandwidth than analog signals. 

For example, Computer signals\\

 Digital instruments are the instruments using the digital signal to indicate the results of measurement in digital form. These devices are more flexible in implementation and draw negligible power. They are expensive than analog devices. Unlike the analog devices these devices are free from observational errors. 
 
 For example, Computers, PDAs e.t.c.

\end{document}